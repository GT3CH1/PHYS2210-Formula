%! Author = gcpease
%! Date = 2/13/2021

\documentclass[11pt]{article}
\usepackage{lingmacros}
\usepackage{tree-dvips}
\usepackage{amsmath}
\usepackage{nccmath}
\usepackage{mathtools}
\usepackage{paracol}
\usepackage{amssymb}
\usepackage[margin=0.2in]{geometry}
\usepackage[T1]{fontenc}
\usepackage{titlesec}
\renewcommand{\baselinestretch}{1}
\titlespacing*{\section}{0pt}{0ex}{0ex}
\titlespacing*{\subsection}{0pt}{0ex}{0ex}
\begin{document}
    \begin{paracol}{3}
        \section*{Constant Acceleration}
        \begin{fleqn}
            $
            d=d_0+v_0t + \frac{1}{2}at^2 \newline
            v_f=v_0+at \newline
            v^2_f = v_0^2 +2a(x-x _0) \newline
            $
        \end{fleqn}
        \section*{Projectile Motion}
        \begin{fleqn}
            $
            v_x(t)=v_{x0} = v_0 \cos \theta_0 \newline
            x(t) = x_0 + v_{x0} + v_o \cos(\theta_i)t \newline
            v_y = y_{y0} -gt = v_0\sin \theta_i -gt \newline
            y(t) = y_0 + v_{y0}t+ \frac{1}{2} at^2 \newline
            y(t) = y_0 +v_0 \sin(\theta_i)t +\frac{1}{2}at^2 \newline
            v^2_y = v_{0y}^2 -2g(\delta y) \newline
            v_f=v_{0y}-gt \newline
            v_f^2 = v_y^2 -2(a)d\newline
            $
        \end{fleqn}
        % Find a better name for this section. I can't remember what its called.
        \section*{Vectors with angles}
        \begin{fleqn}
            $
            A_x=A\cos\theta \newline
            A_y=A\sin\theta \newline
            \tan\theta = \frac{A_y}{A_x} \newline
            \theta = \arctan\frac{A_y}{A_x} \newline
            $
        \end{fleqn}

        \section*{Angular Velocity}
        \begin{fleqn}
            $
            s=r\theta \newline
            v=r\omega \newline
            \omega = \frac{d\theta}{dt} = \frac{2\pi}{T} \newline
            v = \frac{2\pi r}{T} \newline
            T = \frac{2\pi r]}{v} \newline
            $
        \end{fleqn}
        \section*{Kinematics of Constant Angular Acc.}
        \begin{fleqn}
            $
            R = $ radius $ \newline
            \alpha = $ acceleration $ (\frac{rad}{s^2}) \newline
            s=s_0 + v_0t + \frac{1}{2}at^2 \newline
            \theta = \theta_0 + \omega_0 t + \frac{1}{2}\alpha t^2 \equiv \delta \theta = \omega_0 t + \frac{1}{2}\alpha t^2 \newline
            v^2_f = v_0^2+2a(s-s_0)  \newline
            \omega^2_f = \omega_0 ^2 + 2 \alpha (\delta \theta) \newline
            \omega_f = \omega_0 + \alpha t  \newline
            v_f = R\omega \newline
            \delta s = R \delta \theta \newline
            a_c = \frac{v^2_f}{R} $ (centripetal)$\newline
            a_r = \omega^{2}R \newline
            a_t = R \alpha $ (tangential) $ \newline
            v_{ang} = \frac{R}{T} \newline
            a_{ang} = \frac{v_{ang}}{T} \newline
            a_{total} = \sqrt{a_t^2+a_c^2}
            $
        \end{fleqn}
        \switchcolumn
        \section*{Friction}
        \begin{fleqn}
            $
            a=\frac{f_{net}}{m} \newline
            \mu mg = ma \newline
            a= \mu g \newline
            \overrightarrow{F_{net}} = \sum \overrightarrow{F_x}-\overrightarrow{F_{k}} \newline
            \sum F_x = ma = T-f_k \newline
            \sum F_y = +n-mg = 0 \newline
            \sum F_x = F_s - mg\sin\theta \newline
            F_k = \mu_{k}n \newline
            n=mg \newline
            \sum F=mg-F(\cos\theta - \mu_k\sin\theta) \newline
            a = \frac{F-\mu mg}{m} \newline
            $
        \end{fleqn}
        \section*{Newton's Laws}
        \begin{fleqn}
            $
            F=ma \newline
            \sum F_{m_a+m_b} = (m_a+m_b)a \newline
            \sum F = T - mg = ma \newline
            \sum F = \frac{T-F_k}{m} \newline
            \sum F_x=T_1\sin\theta_1 + T_2\sin\theta_2 \newline
            \sum F_y=T_1\cos\theta_1 + T_2\cos\theta_2 \newline
            $
        \end{fleqn}
        \section*{Misc. Equations}
        \begin{fleqn}
            $
            x = \frac{-b \pm \sqrt{b^2-4ac}}{2a}
            $
        \end{fleqn}
    \end{paracol}

\end{document}