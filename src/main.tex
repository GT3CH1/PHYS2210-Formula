%! Author = gcpease
%! Date = 2/13/2021

\documentclass[14pt]{article}
\usepackage{lingmacros}
\usepackage{tree-dvips}
\usepackage{amsmath}
\usepackage{nccmath}
\usepackage{mathtools}
\usepackage{paracol}
\usepackage{amssymb}
\usepackage[margin=0.2in]{geometry}
\usepackage[T1]{fontenc}
\usepackage{titlesec}
\DeclareMathOperator{\arcsec}{arcsec}
\DeclareMathOperator{\arccot}{arccot}
\DeclareMathOperator{\arccsc}{arcsc}
\renewcommand{\baselinestretch}{1.5}
\titlespacing*{\section}{0pt}{0ex}{0ex}
\titlespacing*{\subsection}{0pt}{0ex}{0ex}

\title{PHYS2210 - SP2021 - Formula Card}
\author{Gavin C. Pease}
\begin{document}
    \maketitle
    \begin{paracol}{2}
        \section*{Constant Acceleration}
        \begin{fleqn}
            $
            x=x_0+v_0t + \frac{1}{2}at^2 \newline
            v=v_0+at \newline
            v^2 = v_0^2 +2a(x-x_0) \newline
            $
        \end{fleqn}
        \section*{Free Fall}
        \begin{fleqn}
            $
            y=y_0 +v_{0y} + \frac{1}{2}gt^2 \newline
            v_f=v_{0y}-gt \newline
            v_f^2 = v_y^2 -2y(y-y_0) \newline
            $
        \end{fleqn}
        % Find a better name for this section. I can't remember what its called.
        \section*{Vectors with angles}
        \begin{fleqn}
            $
            A_x=A\cos\theta \newline
            A_y=A\sin\theta \newline
            \tan\theta = \frac{A_y}{A_x} \newline
            \theta = \arctan\frac{A_y}{A_x} \newline
            $
        \end{fleqn}

        \section*{Angular Velocity}
        \begin{fleqn}
            $
            s=r\theta \newline
            v=r\omega \newline
            \omega = \frac{d\theta}{dt} = \frac{2\pi}{T} \newline
            v = \frac{2\pi r}{T} \newline
            T = \frac{2\pi r]}{v} \newline
            $
        \end{fleqn}
        \section*{Kinematics of Constant Angular Acc.}
        \begin{fleqn}
            $
            s=s_0 + v_0t + \frac{1}{2}at^2 \equiv \theta = \theta_0 + \omega_0 t + \frac{1}{2}\alpha t^2 \newline
            v^2 = v_0^2+2a(s-s_0)  \equiv \omega^2 = \omega_0 ^2 + 2 \alpha (\theta - \theta_0)\newline
            a_c = \frac{v^2}{r} \newline
            a_r = \omega^2r \newline
            v_{ang} = \frac{r}{T} \newline
            a_{ang} = \frac{v_{ang}}{T} \newline
            $
        \end{fleqn}
        \switchcolumn
        \section*{Kinetic Friction}
        \begin{fleqn}
            $
            a=\frac{f_{net}}{m} \newline
            \overrightarrow{F_{net}} = \sum \overrightarrow{F} = m \overrightarrow{a} \newline
            F=ma \newline F_k = \mu_{k}n \newline n=mg
            $
        \end{fleqn}
    \end{paracol}

\end{document}