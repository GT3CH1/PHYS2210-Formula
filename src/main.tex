%! Author = gcpease
%! Date = 2/13/2021

\documentclass[11pt]{article}
\usepackage{lingmacros}
\usepackage{tree-dvips}
\usepackage{amsmath}
\usepackage{nccmath}
\usepackage{mathtools}
\usepackage{paracol}
\usepackage{amssymb}
\usepackage[margin=0.2in]{geometry}
\usepackage[T1]{fontenc}
\usepackage{titlesec}
\renewcommand{\baselinestretch}{1}
\titlespacing*{\section}{0pt}{0ex}{0ex}
\titlespacing*{\subsection}{0pt}{0ex}{0ex}
\begin{document}
    \begin{paracol}{3}
        \section*{Constant Acceleration}
        \begin{fleqn}
            $
            d=d_0+v_0t + \frac{1}{2}at^2 \newline
            v=v_0+at \newline
            v^2 = v_0^2 +2a(\Delta x) \newline
            $
        \end{fleqn}
        \section*{Projectile Motion}
        \begin{fleqn}
            $
            v_x(t)=v_{x0} = v_0 \cos \theta_0 \newline
            x(t) = x_0 + v_0 \cos(\theta_i)t \newline
            v_y(t) = v_0\sin\theta_i-gt \newline
            y(t) = y_0 +v_0 \sin(\theta_i)t +\frac{1}{2}at^2 \newline
            v=v_{0y}-gt \newline
            v^2 = v_y^2 -2ad\newline
            $
        \end{fleqn}
        % Find a better name for this section. I can't remember what its called.
        \section*{Vectors with angles}
        \begin{fleqn}
            $
            A_x=A\cos\theta \newline
            A_y=A\sin\theta \newline
            \tan\theta = \frac{A_y}{A_x} \newline
            \theta = \arctan\frac{A_y}{A_x} \newline
            $
        \end{fleqn}

        \section*{Angular Velocity}
        \begin{fleqn}
            $
            s=r\theta \newline
            v=r\omega \newline
            v = \frac{2\pi r}{T} \newline
            \omega = \frac{v}{r} \newline
            \omega = \frac{d\theta}{dt} = \frac{2\pi}{T}\newline
            T = \frac{2\pi r}{v} \newline
            $
        \end{fleqn}
        \section*{Angular Acc.}
        \begin{fleqn}
            $
            r = $ radius $ \newline
            \alpha = $ acceleration $ (\frac{rad}{s^2}) \newline
            \omega = $ velocity $ (\frac{rad}{s}) \newline
            s = $ arc length $ \newline
            s=s_0 + v_0t + \frac{1}{2}at^2 \newline
            s = r \Delta \theta \newline
            \theta = \theta_0 + \omega_0 t + \frac{1}{2}\alpha t^2  \newline
            \Delta \theta = \omega_0 t + \frac{1}{2}\alpha t^2  \newline
            \Delta \theta = \frac{w^2_f - w^2_0}{2 \alpha} \newline
            v_f = r\omega \newline
            v^2_f = v_0^2+2a(\Delta s)  \newline
            \omega^2_f = \omega_0 ^2 + 2 \alpha (\Delta \theta) \newline
            \omega_f = \omega_0 + \alpha \Delta t  \newline
            a_c = \frac{v^2_f}{r} $ (centripetal)$\newline
            a_r = \omega^{2}r \newline
            a_t = r \alpha $ (tangential) $ \newline
            v_{ang} = \frac{r}{T} \newline
            a_{ang} = \frac{v_{ang}}{T} \newline
            a_{total} = \sqrt{a_t^2+a_c^2}
            $
        \end{fleqn}
        \switchcolumn
        \section*{Work/Energy}
        \begin{fleqn}
            $
            K = $ Kinetic Energy$ \newline
            U = $ Potential Energy$ \newline
            $ Master equation $ \newline
            \Delta E^m = -f_k d \newline
            K(U)= \frac{1}{2}mv^2 \newline
            PE = mgd \newline
            W=PE+KE \newline
            $Conservation of Energy$ \newline
            K_1 + U_1 = K_2 + U_2 \newline
            U_{sp} = \frac{1}{2}kx^2 \newline
            $
        \end{fleqn}
        \section*{Momentum}
        \begin{fleqn}
            $
            p = $ Momentum $ \newline
            J = $ Joules $ \newline
            p=mv \newline
            F_{net} = \frac{dp}{dt}, m\frac{v}{dt} \newline
            F_{avg} = \frac{\Delta p}{\Delta t} \newline
            $Impulse = $\int Fdt = J \newline
            $
        \end{fleqn}
        \section*{Collisions/Explosions}
        \begin{fleqn}
            General Eq. (for inelastic, share common final velocity.): \newline
            $
            m_1 v_{1i} + m_2 v_{2i} = m_1 v_{1f} + m_2 v_{2f}
            $
            \begin{itemize}
                \item During inelastic collision, some KE goes to thermal.
                \item If KE is conserved, called perfectly elastic.
                \item Momentum is always conserved
            \end{itemize}
            KE in elastic collision (general, 2D): \newline
            $
            m_1 \frac{v_{1i}^2}{2} + m_2 \frac{v_{2i}^2}{2} = m_1 \frac{v_{1f}^2}{2} + m_2 \frac{v_{2f}^2}{2} \newline
            $
            KE in 1D:
            $
            v_{1i} - v_{2i} = v_{2f} - v_{1f} \newline
            $
            If needed to split to x/y components
            $
            \theta = \tan^{-1}(\frac{v_y}{v_x}) \newline
            v_f = \sqrt{v_y^2 + v_x^2} \newline
            $
        \end{fleqn}
        \section*{Friction}
        \begin{fleqn}
            $
            a=\frac{f_{net}}{m} \newline
            \mu mg = ma \newline
            a= \mu g \newline
            \overrightarrow{F_{net}} = \sum \overrightarrow{F_x}-\overrightarrow{F_{k}} \newline
            \sum F_x = ma = T-f_k \newline
            \sum F_y = n-mg = 0 \newline
            \sum F_x = F_{(s|k)} - mg\sin\theta \newline
            F_k = \mu_{k}mg \newline
            $
        \end{fleqn}
        \switchcolumn
        \section*{Newton's Laws}
        \begin{fleqn}
            $
            F=ma \newline
            \sum F = T - mg = ma \newline
            \sum F = ma + mg \newline
            \sum F = \frac{T-F_k}{m} \newline
            \sum F_{m_a+m_b} = (m_a+m_b)a \newline
            $
        \end{fleqn}
        \section*{Center of Mass}
        \begin{fleqn}
            $
            x_{cm} = \frac{m_1x_1...(+m_2x_2)}{m_1...(+m_2)} \newline
            y_{cm} = \frac{m_1y_1...(+m_2y_2)}{m_1...(+m_2)} \newline
            r_{cm} = \frac1M \sum m_{i}x_i \newline
            K_{rot} = \frac{\omega^2}{2}\sum m_i r_i^2
            $
        \end{fleqn}
        \section*{Misc. Equations}
        \begin{fleqn}
            $
            x = \frac{-b \pm \sqrt{b^2-4ac}}{2a} \newline
            rpm -> rad \frac{rev}{min}\times \frac{2\pi rad \cdot min}{rev \cdot 60s} \newline
            rad -> rev = \frac{rad}{2\pi}
            $
        \end{fleqn}
    \end{paracol}

\end{document}